\documentclass[12pt,a4paper]{article}

\usepackage[utf8]{inputenc}
\usepackage[margin=2.5cm]{geometry}

\usepackage{amsmath,amssymb,amsthm}
\usepackage{commath,mathtools}
\usepackage{parskip}
\usepackage{graphicx}
\usepackage{xcolor}
\usepackage{subcaption}
\usepackage[english]{babel}
\usepackage[bookmarksnumbered]{hyperref}
\usepackage{setspace}
\usepackage{tabularx}


\title{CA Reportage}
\author{
\texttt{r.m.r.dehaas, \hspace{6pt} 
Studentennummer:\hspace{2pt}0954160}
\and
\texttt{f.deVries, \hspace{6pt}
Studentennummer:\hspace{2pt}0881988}}
\date{Y Januari 2024}

\begin{document}
\setstretch{1.2}
\maketitle
\section*{Achtergrond}
Cellulaire automata (CA) is een discreet model. 
Bij een CA is er een regelmatige veld met cellen. 
Elke cel kan in een eindig aantal staten verkeren.
Het simpelste aantal staten is dat cellen 
de staat levend (met binair het getal 1) of dood (met binair het getal 0) kunnen hebben.
Een CA heeft een begin situatie, dus een vast begin met cellen met verschillende staten.
De volgende generatie cellen wordt recursief bepaald. 
De regels van de CA bepalen hoe deze recursie wordt uitgevoerd.
Wij hebben twee specifieke CA's geprogrammeerd, de een-dimensionale en 
twee-dimensionale CA, beide met een vierkante veld. 
De cellen zijn dus vierkant  
We zullen nu de belangrijkste parameters van een CA beschrijven.
De een-dimensionale CA heeft een lengte en de twee-dimensionale een lengte en breedte.
Omdat een computer niet een oneindig groot veld kan maken is het belangrijk dat 
er met de cellen in de grens van het veld iets apart wordt gedaan. Er moet een 'boundary' defintie worden gegeven.
Bij de constante 'boundary' definitie laten we de cellen bij de border constant.
Bij de periodieke 'boundary' definitie wordt de grenzen van het veld als een donut verbonden.
Elke cel heeft buren om zich heen die bepalen wat de nieuwe staat van een cel is.
Bij de twee-dimensionale CA zijn er in ieder geval twee soorten buren mogelijk, de Moore buren en de VonNeumann buren.
Bij de Moore buren worden alle aangrenzende cellen (dus ook diagonaal) als buren gerekend.
De VonNeumann buren is hetzelfde als die van Moore buren behalve dat de buren die diagonaal liggen niet meetellen.
Bij de een-dimensionale CA zijn de linker- en rechterkant van een cell de buren.
Elke CA heeft nog een gedefineerde regel. De regel bepaald wat de nieuwe staat wordt van een cel.
De regels worden recursief uitgevoerd.
Om de regels van de een-dimensionale CA goed te begrijpen kijken we eerst naar het aantal mogelijke combinaties die een cell met zijn buren kan hebben.
Voor een binaire staat zijn de combinaties $111$, $110$, $101$, $100$, $011$, $010$, $001$ en $000$ mogelijk.
Dit zal someteen nodig zijn.
De regel van de CA wordt als nummer geschreven, bijvoorbeeld regel $110$.
Dit nummer wordt als een 8-bits nummer geschreven. 
Elke bit gepresenteerd dan de nieuwe staat die een cel krijgt bij de combinatie die hierboven mogelijk is.
Stel dat we regel $110$ willen gebruiken. Dit is in 8-bits binair $01101110$.
De regels zijn dus:
\begin{center}
\begin{tabular}{|m{0.6cm}|m{0.6cm}|m{0.6cm}|m{0.6cm}|m{0.6cm}|m{0.6cm}|m{0.6cm}|m{0.6cm}|}
    \hline
    $111$ & $110$ & $101$ & $100$ & $011$ & $010$ & $001$ & $000$ \\
    \hline
    $0$ & $1$ & $1$ & $0$ & $1$ & $1$ & $1$ & $0$ \\
    \hline
\end{tabular}
\end{center}
\vspace{10pt}
Dus een levende cell met als buren twee levende cellen is dus als combinatie $111$ en zal in dit geval dus dood gaan (er staat een $0$).
Een bekende regel bij de twee-dimensionale CA is 'the game of life'. 
\newline
De regel van 'the game of life' is als volgt:
\vspace{4pt}
\newline
$\bullet$ Elke levende cel met minder dan twee buren gaat dood (onderpopulatie) 
\newline 
$\bullet$ Elke levende cel met twee of drie buren blijft levend.
\newline
$\bullet$ Elke levende cel met meer dan drie buren gaat dood (overpopulatie)
\newline 
$\bullet$ Elke dode cel met precies drie buren wordt levend (reproductie) 
\newline
\vspace{8pt}
De vraag hierbij is hoe we een CA class kunnen met de een-dimensionale CA en twee-dimensionale CA
waarbij de gebruiker zoveel mogelijk input kan geven zodat de class in vele toepassingen kan worden gebruikt. 


\section*{Implementatie en verantwoording}
Array's zijn voor een veld beter dan lijsten want een veld bevat ten eerste alleen één type (in dit geval het type 'int') 
en array's kunnen ook alleen één type element bevatten.  
Ten tweede veranderen de dimensies van het field niet, en bij array's mogen de dimensies ook niet meer worden veranderd na
het maken van de array. Een dictionary is voor de regels van de een-dimensionale CA heel toepasselijk, want 
de dictionary kan heel mooi geintepeerdeerd en gebruikt worden. 
Namelijk de keys zijn alle mogelijke combinaties die een cell met zijn buren kan hebben
en de values worden door de bits van het binair getal bepaald. 
We proberen daarnaast zoveel mogelijk methodes in de abstracte klassen te zetten.
De een-dimensionale en twee-dimensionale CA hebben bijvoorbeeld de lengte gemeen.   


\section*{Algoritmes}
We hebben een abstracte class CA gemaakt die verschillende 'child' classen heeft.
\newline
De structuur van de klassen zien er als volgt uit:
\newline
\phantom{.}class CA:
\newline
\phantom{.}\hspace{12pt}square CA class:
\newline
\phantom{.}\hspace{24pt}- one-dimensional CA class
\newline
\phantom{.}\hspace{24pt}- two-dimensional CA class
\vspace{6pt}
\newline
De datastructuren die zijn gebruik zijn 'arrays', 'lists' en 'dictionary's'. 
De 'array' is gebruikt voor het veld waarbij de indices de plaats van het element op het veld weergeven en 
de waarde van het element zelf de staat voorstelt.
De dictionary wordt bij de een-dimensionale CA gebruikt voor het opzoeken welke staat een cell moet zijn met de gegeven buren.
De lists was nodig om te lopen over de dictionary.
\newline
Doordat het progamma vrij veel methodes is het moeilijk om 
de tijd complexiteit te bepalen door naar het aantal operaties te kijken.
Makkelijker is om de module time te gebruiken en de verschil van tijd tussen het starten van het progamma en wanneer het progamma klaar is.
Dit geeft een indicatie over tijd complexiteit.  
We hebben bij de een-dimensionale CA en twee-dimensionale CA verschillende combinaties 
inputs gedaan. Elke meting is 3 keer gedaan en hiervan is het gemiddelde genomen.
De gemiddelden van de een-dimensionale CA staan in tabel 1 en die 
van de twee-dimensionale CA in tabel 2. De constante parameters zijn als volgt:
\newline
een-dimensionale CA: (periodic, regel 110)
\newline
twee-dimensionale CA: (Life, Moore, periodic).

\begin{table*}
    \centering
    \begin{tabular}{|m{1.2cm}|m{2.0cm}|m{2.0cm}|m{2.0cm}|}
        \hline
        $\phantom{.}$ & $10$ & $2500$ & $10000$ \\
        \hline
        $10$ & $0.0112$ & $0.0933$ & $0.301$ \\
        \hline
        $100$ & $0.0461$ & $0.493$ & $1.88$ \\
        \hline
        $1000$ & $0.250$ & $4.61$ & $17.6$ \\
        \hline
    \end{tabular}
    \caption{The tijd complexiteit van de een-dimensionale CA}
    \label{tbl:1dim}
\end{table*}

\begin{table*}
    \centering
    \begin{tabular}{|m{1.2cm}|m{2.0cm}|m{2.0cm}|m{2.0cm}|}
        \hline
        $\phantom{.}$ & $10x10$ & $50x50$ & $100x100$ \\
        \hline
        $10$ & $0.0150$ & $0.114$ & $0.373$ \\
        \hline
        $100$ & $0.0820$ & $0.708$ & $2.68$ \\
        \hline
        $1000$ & $0.652$ & $6.51$ & $25.5$ \\
        \hline
    \end{tabular}
    \caption{The tijd complexiteit van de twee-dimensionale CA}
    \label{tbl:2dim}
\end{table*}

\section*{Code}
Een link voor de code is hier: 
\textcolor{blue}{https://github.com/Anophelesgambiae/PinW-Cellulair-automata-project}
\newline
Bij het opstarten van het progamma wordt er van de gebruiker input gevraagd.
\newline
Bij zowel de een-dimensionale als twee-dimensionale CA moet 
er een lengte en 'boundary' conditie als input worden gegeven.
Daarnaast moet er in de een-dimensionale CA een regel nummer worden gegeven.
Bij de twee-dimensionale CA moet ook een regel worden gegeven als input, maar niet als nummer (er zijn teveel mogelijkheden).
Een input kan zijn life. De regels van 'the game of life' worden dan gebruikt.
Verder moet er nog een burenregel als input worden gegeven (Moore of VonNeumann).

\section*{Resultaten}
Dit zijn de patronen die deze CA's kunnen maken:
\newline
\graphicspath{ {./CA_project_2024/} }
\includegraphics[scale=0.6]{1dimCA_100__rule110_periodic_100}
\includegraphics[scale=0.6]{1dimCA_100__rule30_constant_100}
\includegraphics[scale=0.6]{2dimCA_100x100__Life_Moore_periodic_100}
\includegraphics[scale=0.6]{2dimCA_100x100__Life_vonNeumann_periodic_100}

\section*{Conclusie}
\textcolor{red}{Moet nog worden gedaan}

\section*{Discussie}
Er is alleen de een-dimensionale CA en twee-dimensionale 
CA gemaakt, maar er bestaat natuurlijk ook de drie-dimensionale CA.
Voor toekomstige versies zou een drie-dimensionale CA een goede uitbreiding zijn.
Daarnaast zou de CA ook uitgebreid kunnen worden met CA waar de cellen meer dan een binaire staat hebben, maar wel eindig.
Ook kan er CA worden geïnplementeerd waar de cellen niet vierkant zijn, maar bijvoorbeeld hexagonaal.
Kortom, onze code is alleen goed voor een heel beperkt aantal CA's.
Als laatste hadden sommige methodes in de abstracte klassen moeten worden gezet, maar het 
is makkelijker om het eerst in de lagere klassen te stoppen.
Voor toekomstige versies zal dit moeten worden geïmplementeerd.  

\newpage
\section*{Takenverdeling}
We hebben de taken als volgt de verdeeld:
\vspace{6pt}
\newline
Robert de Haas:
\newline
1) abstracte CA klassen
\newline
2) twee-dimensionale CA klasse
\newline
3) report maken
\newline
4) complexiteit meten
\vspace{6pt}
\newline
Fabian de Vries:
\newline
1) een-dimensionale CA klasse
\newline
2) het bekijken van het visueel maken van de CA
newline
3) Handleiding 


We hebben wel de code (dus de twee klassen) zo analoog mogelijk gemaakt.

  

\end{document}