\documentclass[12pt,a4paper]{article}

\usepackage[utf8]{inputenc}
\usepackage[margin=2.5cm]{geometry}

\usepackage{amsmath,amssymb,amsthm}
\usepackage{commath,mathtools}
\usepackage{parskip}
\usepackage{graphicx}
\usepackage{xcolor}
\usepackage{subcaption}
\usepackage[english]{babel}
\usepackage[bookmarksnumbered]{hyperref}
\usepackage{setspace}


\title{CA Reportage}
\author{
\texttt{r.m.r.dehaas, \hspace{6pt} 
Studentennummer:\hspace{2pt}0954160}
\and
\texttt{Fabian.deVries, \hspace{6pt}
Studentennummer:\hspace{2pt}X}}
\date{Y Januari 2024}

\begin{document}
\setstretch{1.2}
\maketitle
\section*{Achtergrond}
Cellulaire automata (CA) is een discreet model. 
Bij een CA is er een regelmatige veld met cellen. 
Elke cel kan in een eindig aantal staten verkeren.
Het simpelste aantal staten is dat cellen 
de staat levend (met binair het getal 1) of dood (met binair het getal 0) kunnen hebben.
Een CA heeft een begin situatie, dus een vast begin met cellen met verschillende staten.
De volgende generatie cellen wordt recursief bepaald. 
De regels van de CA bepalen hoe deze recursie wordt uitgevoerd.
Wij hebben twee specifieke CA's geprogrammeerd, de een-dimensionale en 
twee-dimensionale CA, beide met een vierkante veld. 
De cellen zijn dus vierkant  
We zullen nu de belangrijkste parameters van een CA beschrijven.
De een-dimensionale CA heeft een lengte en de twee-dimensionale een lengte en breedte.
Omdat een computer niet een oneindig groot veld kan maken is het belangrijk dat 
er met de cellen in de grens van het veld iets apart wordt gedaan. Er moet een 'boundary' defintie worden gegeven.
Bij de constante 'boundary' definitie laten we de cellen bij de border constant.
Bij de periodieke 'boundary' definitie wordt de grenzen van het veld als een donut verbonden.
Elke cel heeft buren om zich heen die bepalen wat de nieuwe staat van een cel is.
Bij de twee-dimensionale CA zijn er in ieder geval twee soorten buren mogelijk, de Moore buren en de VonNeumann buren.
Bij de Moore buren worden alle aangrenzende cellen (dus ook diagonaal) als buren gerekend.
De VonNeumann buren is hetzelfde als die van Moore buren behalve dat de buren die diagonaal liggen niet meetellen.
Bij de een-dimensionale CA zijn de linker- en rechterkant van een cell de buren.
Elke CA heeft nog een gedefineerde regel. De regel bepaald wat de nieuwe staat wordt van een cel.
De regels worden recursief uitgevoerd.
\textcolor{red}{Hulp nodig bij de uitleg van de regels van de 1d CA}
\newpage
Een bekende regel bij de twee-dimensionale CA is 'the game of life'. 
\newline
De regel van 'the game of life' zijn als volgt:
\vspace{4pt}
\newline
$\bullet$ Elke levende cel met minder dan twee buren gaat dood (onderpopulatie) 
\newline 
$\bullet$ Elke levende cel met twee of drie buren blijft levend.
\newline
$\bullet$ Elke levende cel met meer dan drie buren gaat dood (overpopulatie)
\newline 
$\bullet$ Elke dode cel met precies drie buren wordt levend (reproductie) 
\newline
\vspace{4pt}
\textcolor{red}{De vraag en rest moet nog geschreven worden}


\section*{Implementatie en verantwoording}
\textcolor{red}{Moet nog geschreven worden}


\section*{Algoritmes}
\textcolor{red}{Nog iets met design van de allgoritme}
De datastructuren die zijn gebruik zijn 'arrays', 'lists' en 'dictionary's'.

\textcolor{red}{Nog iets met complexity} 

\section*{Code}
\textcolor{red}{De manier om input te geven moeten we nog bespreken}
\newline
Bij zowel de een-dimensionale als twee-dimensionale CA moet 
er een lengte en 'boundary' conditie als input worden gegeven.
Daarnaast moet er in de een-dimensionale CA een regel nummer worden gegeven.
Bij de twee-dimensionale CA moet ook een regel worden gegeven als input, maar niet als nummer (er zijn teveel mogelijkheden).
Een input kan zijn life. De regels van 'the game of life' worden dan gebruikt.
Verder moet er nog een burenregel als input worden gegeven (Mooren of VonNeumann).


\section*{Resultaten}
\textcolor{red}{Moet nog worden gedaan}

\section*{Conclusie}
\textcolor{red}{Moet nog worden gedaan}

\section*{Discussie}
\textcolor{red}{Moet nog worden gedaan}

\section*{Takenverdeling}
We hebben de taken als volgt de verdeeld:
\vspace{6pt}
\newline
de Haas:
\newline
1) abstracte CA klassen
\newline
2) twee-dimensionale CA klasse
\vspace{6pt}
\newline
de Vries:
\newline
1) een-dimensionale CA klasse
\newline
\textcolor{red}{volgt meer}

We hebben wel de code (dus de twee klassen) zo analoog mogelijk gemaakt.

  

\end{document}